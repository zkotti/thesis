\section{Preface}
\label{sec:preface}

\hfill

\begin{quote}
\centering
\textit{
``Romance should never begin with sentiment. \\
It should begin with science and end with a settlement.''
}
\end{quote}
\hfill --- Oscar Wilde, An Ideal Husband

\hfill

\hfill

\emph{What other people think} has always been an important piece of information
for most of us during the decision-making process.
Long before awareness of the World Wide Web became widespread,
many of us asked our friends to recommend an auto mechanic
or to explain who they were planning to vote for in local elections,
requested reference letters regarding job applicants from colleagues,
or consulted Consumer Reports to decide what dishwasher to buy.

But the Internet and the Web have now (among other things) made it possible
to find out about the opinions and experiences of those in the vast pool of people that are neither our personal acquaintances
nor well-known professional critics --
that is, people we have never heard of.
And conversely, more and more people are making their opinions available
to strangers via the Internet~\cite{PL08}.

The interest that individual users show in online opinions
about products and services,
and the potential influence such opinions wield,
is something that vendors of these items are paying more and more attention to~\cite{Hof08}.
The following excerpt from a white paper is illustrative
of the envisioned possibilities,
or at the least the rhetoric surrounding the possibilities:

\begin{quote}
``With the explosion of Web 2.0 platforms
such as blogs, discussion forums, peer-to-peer networks,
and various other types of social media \dots
consumers have at their disposal a soapbox
of unprecedented reach and power
by which to share their brand experiences and opinions,
positive or negative, regarding any product or service.
As major companies are increasingly coming to realize,
these consumer voices can wield enormous influence
in shaping the opinions of other consumers --
and, ultimately, their brand loyalties, their purchase decisions,
and their own brand advocacy. \dots
Companies can respond to the consumer insights they generate
through social media monitoring and analysis
by modifying their marketing messages, brand positioning, product development,
and other activities accordingly.''

\hfill --- Zabin and Jefferies~\cite{ZJ08}
\end{quote}

But industry analysts note that the leveraging of new media for the
purpose of tracking product image requires new technologies; here is a
representative snippet describing their concerns:

\begin{quote}
``Marketers have always needed to monitor media
for information related to their brands --
whether it's for public relations activities,
fraud violations, or competitive intelligence.
But fragmenting media and changing consumer behaviour
have crippled traditional monitoring methods.
Technorati estimates that 75,000 new blogs are created daily,
along with 1.2 million new posts each day,
many discussing consumer opinions on products and services.
Tactics [of the traditional sort]
such as clipping services, field agents,
and ad hoc research simply can't keep pace.''

\hfill --- Kim~\cite{Kim06}
\end{quote}

Thus, aside from individuals, an additional audience for systems capable
of automatically analysing consumer sentiment, as expressed in no
small part in online venues, are companies anxious to understand how
their products and services are perceived.

The field of Sentiment Analysis aims to satisfy the aforementioned need.
In the following sections a thorough description of sentiment analysis is provided,
along with methodologies for its application to text data.
A segregation is performed
among the different approaches used in sentiment analysis,
as well as between multilingual and Greek applications.
Finally, a specific methodology is suggested for the implementation
of lexicon-based sentiment analysis on Greek text data.

The research questions that this thesis attempts to answer are the following.

\begin{enumerate}
 \item \emph{What levels, features and approaches are used in sentiment analysis?} \\
 This is answered in Section~\ref{sec:analysis}.
 \item \emph{What are the advantages of lexicon-based sentiment analysis?
 What related work exists in literature?
 How has the Greek language been studied?} \\
 This is answered in Section~\ref{sec:lexicon}.
 \item \emph{What methodology for sentiment analysis of Greek text data is suggested in this thesis?} \\
 This is answered in Section~\ref{sec:methodology}.
\end{enumerate}
