\section{Conclusions}
\label{sec:conclusions}

isMOOD is a fast-growing and competitive company in the field of Sentiment Analysis.
It started as a start-up during the peak of the Greek financial crisis,
and managed not only to endure but also to expand
by obtaining a large portion of the Greek market.
Nowadays isMOOD is making its first steps into the Cypriot market as well.

Working at isMOOD was a pleasant activity.
The environment was great and all my colleagues were cheerful
and always willing to provide assistance to each other.
Two significant advantages of the company 
that avail its overall performance are its ability to adapt to new challenges,
and the diversity of the characters of the employees.
The first is reinforced by the fact
that I had the opportunity to experiment in both teams,
and change from the Business to the Technical upon my request.
The second advantage stems from the fact
that one employee is great at communication,
another is talented at coordination,
a third is capable of remaining calm and rational
even at the most stressful times.

The Technical team where I was positioned is consisted of two people.
The cooperation of these two people is admirable as one complements the other.
They both provided me with a variety of knowledge, skills and best practices.
Furthermore, their commitment, passion and diligence on programming
was a continuous motivation for me to work harder.

Beyond the goals of the project I was engaged in,
my colleagues of the Technical team were always willing
to train me on practices and tools that I was not familiar with.
For instance, they taught me how to work with Virtual Machines,
screen sessions and FileZilla.
I improved my code in Python both in terms of style by using PEP 8,
and in terms of performance.
I was trained for the first time on a NoSQL database, MongoDB.
I managed to build a database in MongoDB through Python.
I was introduced to an alternative of GitHub, GitLab.

Finally, a significant experience I acquired through my internship
was the continuous sense of responsibility.
Despite having previously worked on multiple and diverse University projects,
it was the particular project I implemented at isMOOD
that transfered me the great sense of responsibility
due to the fact that, for the first time,
a project I worked on would be evaluated in production.
Knowing that your work will be assessed not only by your supervisor
but also by clients, and have an actual impact on them,
provides a different perspective on your project.
