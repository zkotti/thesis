\section{Internship Description}
\label{sec:internship}

\subsection{Department}
\label{subsec:department}

I was initially assigned to the Business team of isMOOD
as a Business Developer.
After a couple of days and a lot of interaction
with both the Business and the Technical team,
I realised I was more suitable
and more interested in the technical activities
of the company.
This led me to a thorough discussion
initially with my supervisor of the Business team, Georgina Gkika,
and then with the Managing Director of the company,
Anna Kasimati.

Thankfully, they were both eager
to help me experiment in the Technical team as well,
and then decide together in which team
I would finally be assigned to.
Thus, another discussion took place
with the Lead Software Engineer of the Technical team,
Stauros Triantafyllos.
I was informed about a new project
that the company wanted to initialise
regarding the implementation of an algorithm in Python
for lexicon-based sentiment analysis of Greek text data.

I spent a two-week time in the Technical team on trial,
and then had a final discussion with Anna, Stauros, and Georgina,
and we all decided that I was more appropriate for the Technical team.

The Technical team is composed of two people:
Stauros Triantafyllos, the Lead Software Engineer,
and Panagiotis Plytas, the Front-end Engineer.
The purpose of the team is to provide support
for everything related to the activities of isMOOD.
They are responsible for the development, maintenance,
and security of the isMOOD platform,
for the advancement of the technical approaches in use,
the amelioration of the algorithms,
the data storage, maintentance and security,
the resolution of any technical issue that occurs
to both a client or isMOOD itself,
the facilitation of the employees in their regular activities,
and for the overall contentment of the clients.

\subsection{Position}
\label{subsec:position}

The position I was finally assigned to was of the Back-end Developer.
As mentioned above, I was requested
to implement a particular project regarding sentiment analysis.
This project involved first the construction 
of a database in MongoDB,
and then the development of an algorithm in Python
for the application of sentiment analysis on Greek text data.

\subsection{Required Skills}
\label{subsec:skills}

In order to fulfil the goals of the project,
three skills were required.
The first skill involves basic familiarity with research conduction.
In order to explore best practices and approaches
related to the topic of the project I was assigned,
I had to be familiar with research performance,
particularly digital libraries where research is maintained,
and how to properly study and evaluate research.

The second skill involves experience with databases,
especially NoSQL databases.
Despite having previous experience only with SQL databases,
I managed to learn the basic operation of MongoDB in a few days.

The third skill required for the implementation of the algorithm
is experience with Python.
I already had sufficient knowledge of Python
from University courses,
thus this project allowed me to build and expand on my previous knowledge
by exploring new libraries and better programming practices.

Additional skills that were also beneficial to my internship
are collaboration, ease of communication, and problem solving.
Working on multiple group projects at University
has contributed to the development of these three critical attributes.

\subsection{Expected Results}
\label{subsec:results}

The expected results of my internship were
the successful construction of the database in MongoDB,
and the development of the algorithm in Python.
Apart from these, it was my personal intention
to satisfy the expectations of my supervisor at isMOOD,
and also to manage favourable collaboration with my colleagues.
