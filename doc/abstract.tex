\addcontentsline{toc}{chapter}{Abstract}
\begin{center}
\textbf{\large Abstract}
\end{center}

An important part of our information-gathering behaviour has always been
to find out what other people think.
With the growing availability and popularity of opinion-rich resources
such as online review sites, personal blogs and social media channels,
new opportunities and challenges arise
as people now can, and do, actively use information technologies
to seek out and understand the opinions of others.
The sudden eruption of activity
in the area of opinion mining and sentiment analysis,
which deals with the computational treatment of opinion,
sentiment, and subjectivity in text,
has thus occurred at least in part
as a direct response to the surge of interest in new systems
that deal directly with opinions as a first-class object.

This thesis aims to provide a literature review on sentiment analysis,
the widespread available methods for its application to text data,
with an emphasis on lexicon-based approaches for the Greek language.
In addition, a particular methodology is proposed and implemented
in the context of an internship,
for the application of lexicon-based sentiment analysis to Greek text data,
which facilitates from a well-reputed English sentiment lexicon
by translating it into Greek and transferring sentiment to a Greek lexicon.

