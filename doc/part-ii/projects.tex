\section{Projects/Activities}
\label{sec:projects}

\subsection{Business team}
\label{subsec:business-activities}

During the short time that I was part
of the Business team as a Business Developer,
I worked on the verification of the sentiment
that is appended to all social media text data
of projects for Nestle, a marketing company, National Bank of Greece,
Bank of Cyprus, and a few supermarkets.
I was also trained on the customised reports
that are provided to Nestle and National Bank of Greece,
and had the opportunity
to create on my own a daily report for National Bank of Greece
that was then sent to them.

In addition to the reports,
I was trained on testing potential new projects
for existing or new clients,
and on officially creating them on the platform.
Before a new project is created at the isMOOD platform,
the Business team collects all available data from social media
related to this project
by using relevant keywords.
This process is performed with Python scripts.
If there are sufficient data available
to start a new project,
then the team creates a new project on the platform
through another Python script.

\subsection{Technical team}
\label{subsec:technical-activities}

When I joined the Technical team as a Back-end Developer,
I was assigned a project regarding the implementation of an algorithm
for lexicon-based sentiment analysis on Greek text data.
This project involves two phases:
the construction of a database in MongoDB
for the Greek sentiment lexicon,
and the development in Python of the algorithm
that uses this lexicon
in order to calculate the overall sentiment of a particular Greek text.

In order to explore the most efficient technologies and tools
that would be used in both the construction of the database
and the development of the algorithm,
and also in order to familiarise with the field of sentiment analysis,
I devoted a two-week time to study related published research.

Since not a lot of research has been conducted
on the particular topic for the Greek language,
I had to develop a new approach based on existing literature.
For that purpose, I consulted a few specialists
whose advice was valuable and much appreciated;
my academic advisor, Prof. Diomidis Spinellis,
my company supervisor, Stauros Triantafyllos,
Assoc. Prof. Panos Louridas and Konstantina Dritsa,
who have previously worked on applying sentiment
to the Hellenic Parliament proceedings~\cite{Dri18},
and Marios Papachristou,
who provided me with an extra source of Greek terms
for the lexicon I had to create,
and with a Python lemmatisation library for the Greek language
(further explained in Section~\ref{sec:results}).

By combining all the suggestions and knowledge I gathered
from the research I performed and the professionals I consulted,
I developed in collaboration with my supervisor
the approach that we would follow
for the accomplishment of the project.

Apart from this project,
during my internship I was often requested
to provide my thoughts and opinion on various subjects.
For instance, when a new feature of the platform was developed,
I would usually review and evaluate it with my colleagues.
In addition, when a colleague of the Business team
was in need for assistance with his tasks,
for instance with the verification of the sentiment appended
to the text data of a project,
if I was available, I would always help.
