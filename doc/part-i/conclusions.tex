\section{Conclusions}
\label{sec:conclusions1}

Automated opinion and/or sentiment mining is a very promising topic
with potential applications in social, political, marketing, financial,
and other fields.
With the abrupt emergence of the social media use 
a lot of information is now available regarding user opinions
that can be used to extract valuable insights.
Companies or other stakeholders can benefit from such insights
in order to improve their services and overall performance.

Various attempts have been made for the analysis of sentiments
embedded in users' documents, posts, comments, artefacts
by the research community.
Some focus on specific parts of a document such as sentences
or entities and aspects, while others evaluate the document as a whole.
There are also works that integrate particular features into the analysis
such as semantic or syntactic features.
Two are the most frequently encountered approaches to perform sentiment analysis.
The first involves machine learning methods
whereas the second incorporates the use of sentiment lexicons.
Related work has demonstrated that the machine learning approach performs better
in domain-specific analyses, as opposed to the lexicon-based approach
which is more preferable for cross-domain analyses.

A variety of methods have been proposed in literature
for the construction of accurate and adequate sentiment lexicons.
Some suggest manual annotation of a lexicon
while others propose more automated approaches
using for instance a set of seed words and online dictionaries
for the propagation of the seed list with synonyms and antonyms.
Linguistic rules are often applied to lexicon-based sentiment analyses
such as part-of-speech information, connectives, or contextual valence shifters.

Another important aspect of the lexicon-based approach is the word identification.
In order to match a word of a text to a term of the sentiment lexicon,
three main methods are widely used.
The first proposes stemming of the word but with a threat of over-stemming
or under-stemming. The second suggests lemmatisation,
and the third employs the concept of edit distance.
Edit distance is connected to some sort of bias
resulted from the manual selection of thresholds.
For instance, Dritsa~\cite{Dri18} chose to use a threshold of 0.97
when she implemented the Jaro-Winkler algorithm.

The Greek language has been a challenge for a lot of researchers
mainly due to the lack of publicly available Greek sentiment lexicons.
In addition, the fact that it is a highly inflected language
with multiple variants for each term complicates the process
of word identification.
Therefore, sentiment analysis for the Greek language, particularly
lexicon-based, is an active and promising area of research
and we are still on the foothills of it.

Finally, in this thesis a methodology is suggested for the construction
of a Greek sentiment lexicon based on the translation of an English,
along with a baseline algorithm for the sentiment analysis of Greek documents
inspired from the aggregate-and-average method.
This methodology is implemented and evaluated in Part II of this artefact
in the context of an internship.
